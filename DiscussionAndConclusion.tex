\vspace{-0.5cm}

\section{Discussion And Conclusion} \label{sec:discussion}
    This work proposes a Bayesian \gls{NN}-based approach to estimate the \gls{AIF} from dynamic \gls{PET} images and clinical variables (i.e. age, sex, \gls{TSPO} genotype, injected dose, clinical population). This approach has similarities with with previous methods developed for \gls{PET} tracers, who do not produce radio-metabolites, such as [$^{18}$F]\gls{FDG}. Here, additional efforts were devoted to deal with \gls{PBR28} radio-metabolite correction, as population-based approaches cannot account for between-subjects variability (cite). Moreover, the proposed method seeks to estimate not only a signal but also its uncertainty, which can provide a measure of confidence in the generated signal for unseen data, as well as be used in further computations. Additionally, not only this method extracts a signal and metabolite-corrects it, but because the method is split in multiple parts, each part can be used independently; for instance, here where metabolite correction was applied to a signal generated by a more traditional method (\gls{IDIF}).

    The proposed candidate signals, seen in~\Fref{sec:candidates}, were compared to the gold standard TRUE-\gls{AIF}, obtained from arterial blood sampling and metabolite correction, see~\Fref{sec:dataproc}. Overall, \gls{NN}-\gls{AIF} outperformed all the candidate signals, as expected, in terms of correlation to the ground truth \gls{VT}, see~\Fref{fig:correlation}. However, while \gls{NN}-\gls{AE}\gls{IF} showed comparable performance in terms of correlation and bias to \gls{NN}-\gls{AIF}, it also showed the lowest variance of the estimated \gls{VT}, as measured by the \gls{CV}, see~\Fref{fig:CVar}. This can potentially be explained by the amount of input data being larger and thus the model itself having more parameters.
    
    Interestingly, \gls{NN}-\gls{IDIF} was able to improve on the \gls{IDIF} approach, as demonstrated by a higher correlation coefficient and lower angular distance from the identity line. Overall, \gls{NN}-\gls{IDIF} showed a similar variance to \gls{NN}-\gls{AIF} in the estimate, possibly due to the fact that they share exactly the same method (differing only in the input signal). Thus, the variance seen is probably mostly due to the accuracy of the metabolite correction estimated from the clinical features, see~\Fref{sec:NNDesign}.
    
    The proposed approach has several limitations, including the small training size which did not allow to assess the accuracy of the prediction within subsets of clinical populations in the test-set (i.e. patients vs healthy controls). It is possible that the tracer metabolism will show higher between-subjects variability in patient populations from different clinical conditions, as compared to healthy subjects. Moreover, while this method was only tested on a pool of chronic pain subjects and healthy individuals, within an age rage of $\approx 55 \pm 16$ years, the accuracy of the predicted \glspl{VT} on a larger variety of pathological conditions (i.e. \gls{ALS}, schizophrenia) and on a wider age-range should be evaluated, as both factors are known to significantly affect the radiotracer metabolism (ref). 
    
    It is important to mention that \gls{NN}-\gls{AIF} has a significantly improved computation time over \gls{NN}-\gls{IDIF}; this is due to the fact that, once the model is trained, it is fast to execute again, as compared to the computation time of \gls{IDIF}. On the other hand, each time \gls{NN}-\gls{AIF} is to be applied to a new tracer or scanner, it will need to be retrained.

    Overall, while \gls{VT} computed via Logan graphical method was used as main parameter of interest in this work, the possibility of accurately estimating \gls{PET} kinetic micro-parameters (K$1$, k$2$, etc.) from \glspl{TCM} remains to be determined. It is possible that the performances of \gls{NN}-\gls{AIF} and \gls{NN}-\gls{IDIF} may be further differentiated when looking at those individual parameters.
    
    In the future, the accuracy of the model could be improved through the inclusion of an attention layer either before or after the latent layer of the \gls{AE}, validated through the use of an ablation study. Furthermore, the accuracy of the signal extraction method could be improved by respecting the temporal nature of the data more. The convolutional parts of the network are time-distributed, however the fully connected layers currently used can mix information between time points indiscriminately; \gls{LSTM} layers were tried but found to be less robust than fully connected layers. It is possible that transformer encoder layers, as seen in \gls{NLP}, could respect the temporal nature of the data while also being robust; future work will be needed to corroborate this hypothesis.
