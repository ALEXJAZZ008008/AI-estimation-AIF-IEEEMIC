\vspace{-0.5cm}

\section{Discussion And Conclusion} \label{sec:discussion}
    This work proposes a Bayesian \gls{NN}-based approach to estimate the \gls{AIF} from dynamic \gls{PET} images and clinical variables (i.e. age, sex, \gls{TSPO} genotype, injected dose, clinical population). This approach has similarities with previous methods developed for \gls{PET} tracers that do not produce radio-metabolites, such as [$^{18}$F]\gls{FDG}. Here, additional efforts were devoted to deal with \gls{PBR28} radio-metabolite correction, as population-based approaches cannot account for between-subjects variability~\cite{Mertens2021MinimallyFunction}. Moreover, the proposed method seeks to estimate not only a signal but also its uncertainty, which can provide a measure of confidence in the generated signal for unseen data, as well as be used in further computations. Additionally, not only this method extracts a signal and metabolite-corrects it, but because the method is split in multiple parts, each part can be used independently; for instance, here where metabolite correction was applied to a signal generated by a more traditional method (\gls{IDIF}).  The four candidate signals, seen in~\Fref{sec:candidates}, were compared to the gold standard TRUE-\gls{AIF}, obtained from arterial blood sampling and metabolite correction. Overall, \gls{NN}-\gls{AE}\gls{IF} showed comparable performance in terms of correlation and bias to \gls{NN}-\gls{AIF}, with the lowest variance of the estimated \gls{VT}, as measured by the \gls{CV}. This can potentially be explained by the amount of input data being larger and thus the model itself having more parameters.
    
    Interestingly, \gls{NN}-\gls{IDIF} was able to improve on the \gls{IDIF} approach, as demonstrated by a higher correlation coefficient and lower angular distance from the identity line. 
    
    The proposed approach has several limitations, including the small training size which did not allow to assess the accuracy of the prediction within subsets of clinical populations in the test-set (i.e. patients vs healthy controls). 
    
    In the future, the accuracy of the model could be improved through the inclusion of an attention layer either before or after the latent layer of the \gls{AE}, validated through the use of an ablation study. 
