\vspace{-0.5cm}

\section{Introduction} \label{sec:introduction}
    \IEEEPARstart{T}{he} \gls{VT} estimated with an \gls{AIF} is the gold standard for \gls{PET} quantification of \gls{PBR28} binding. This quantification, however, requires the concurrent measurement of the concentrations of unchanged radioligand in arterial plasma. Although insertion of an arterial catheter rarely results in clinically relevant adverse events~\cite{Everett2009SafetySubjects}, it is laborious, invasive and often discourages subjects from volunteering for \gls{PET} studies. \gls{IDIF} represents a promising alternative to arterial sampling, consisting of the extraction of the input signal from a blood pool visible on \gls{PET} dynamic images, such as the internal carotid arteries/siphons in the case of brain \gls{PET} studies~\cite{Zanotti-Fregonara2011}. However, because of coarse \gls{PET} framing and partial volume effects, \gls{IDIF} typically shows inaccuracies in the estimation of both the shape and amplitude of the \gls{IF}. Moreover, \gls{IDIF} does not allow for the possibility of distinguishing the tracer parent compound from its radio-metabolites, which would require either a prior knowledge or in vitro analysis of blood samples. These aspects limit the applicability of \gls{IDIF} in quantitative \gls{PBR28} \gls{PET} studies.
    
    Recently, the possibility to utilise \gls{ML} to predict the \gls{AIF} from \gls{PET} images has been explored~\cite{Kuttner2020, Wang2020DirectImaging, Ferrante2022PhysicallyImaging}. While those methods have shown promising results, the vast majority of those approaches have been developed for \gls{PET} tracers that do not have radio-metabolites. Moreover, existing literature relies on less sophisticated \gls{NN}-based approaches where the uncertainty of the prediction is not taken into account.
    
    Here, we propose a \gls{NN}-based method for extracting the metabolite corrected \gls{AIF}, that consists of a feature extraction \gls{AE}, a signal extraction network, and a metabolite correction network. Specifically for the \gls{AE}, although also present in the other networks, we try to enforce the low dimensional representation of the input data as disentangled and continuous. Furthermore, the network does not predict a single signal for each input; rather, it predicts a probability density function of potential signals, which allows for the estimation of uncertainty of the model's output.
    
    % it can be applied retrospectively, it isn't prone to user error, it can be used to save scans where the blood has failed, sites require blood sampling facilities and staff
