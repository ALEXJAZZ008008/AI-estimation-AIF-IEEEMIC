\documentclass{IEEEtran}

\usepackage{amsmath}
\usepackage{amssymb}
\usepackage{amsfonts}

\ifCLASSINFOpdf
   \usepackage[pdftex]{graphicx}
\else
   \usepackage[dvips]{graphicx}
\fi

\ifCLASSOPTIONcompsoc
  \usepackage[caption=false, font=normalsize, labelfont=sf, textfont=sf]{subfig}
\else
  \usepackage[caption=false, font=footnotesize]{subfig}
\fi

\usepackage{textcomp}
\usepackage{nicefrac}
\usepackage{siunitx}
\usepackage{fancyref}

\usepackage[style=ieee, doi=false, isbn=false, url=false, maxbibnames=1, minbibnames=1, maxcitenames=1, mincitenames=1, backend=biber, defernumbers=false]{biblatex}
\addbibresource{./references.bib}

\AtEveryBibitem{\clearfield{month}}
\AtEveryBibitem{\clearfield{day}}
\AtEveryBibitem{\clearfield{volume}}
\AtEveryBibitem{\clearfield{issue}}
\AtEveryBibitem{\clearfield{pages}}
\AtEveryBibitem{\clearfield{number}}
\AtEveryBibitem{\clearfield{title}}
\AtEveryBibitem{\clearfield{isbn}}
\AtEveryBibitem{\clearfield{keywords}}
\AtEveryBibitem{\clearfield{issn}}
\AtEveryBibitem{\clearfield{journal}}

\usepackage[activate={true, nocompatibility}, final, tracking=true, kerning=true, spacing=true, factor=1100, stretch=10, shrink=10]{microtype}
\linespread{0.9}

\usepackage{glossaries}

\newacronym{AI}{AI}{Artificial Intelligence}
\newacronym{AE}{AE}{Auto-Encoder}
\newacronym{AC}{AC}{Attenuation Correction}
\newacronym{ALS}{ALS}{Amyotrophic Lateral Sclerosis}
\newacronym{GPU}{GPU}{Graphics Processing Unit}
\newacronym{FDG}{FDG}{fluorodeoxyglucose}
\newacronym{AIF}{AIF}{Arterial Input Function}
\newacronym{AUC}{AUC}{Area Under the Curve}
\newacronym{cLBP}{cLBP}{chronic Low Back Pain}
\newacronym{CV}{nCV}{Normalised Coefficient of Variance}
\newacronym{MSE}{MSE}{Mean Squared Error}
\newacronym{MR}{MR}{Magnetic Resonance}
\newacronym{PBR28}{[$^{11}$C]-PBR$28$}{[$^{11}$C]-Peripheral Benzodiazepine Receptor}
\newacronym{MNI}{MNI}{Montreal Neurological Institute}
\newacronym{FWHM}{FWHM}{Full Width Half Maximum}
\newacronym{HC}{HC}{Healthy Control}
\newacronym{HAB}{HAB}{High Affinity Binder}
\newacronym{MAB}{MAB}{Mixed Affinity Binder}
\newacronym{PT}{PT}{Patient}
\newacronym{HPLC}{HPLC}{High Performance Liquid Chromatography}
\newacronym{IDIF}{IDIF}{Image Derived Input Function}
\newacronym{LL}{LL}{Log-Likelihood}
\newacronym{LSTM}{LSTM}{Long Short Term Memory}
\newacronym{KOA}{KOA}{Knee Osteo-Arthritis}
\newacronym{MAE}{MAE}{Mean Absolute Error}
\newacronym{ML}{ML}{Machine Learning}
\newacronym{NN}{NN}{Neural Network}
\newacronym{SUV}{SUV}{Standardised Uptake Value}
\newacronym{TAC}{TAC}{Time Activity Curve}
\newacronym{OSEM}{OSEM}{Ordered Subset Expectation Maximisation}
\newacronym{PET}{PET}{Positron Emission Tomography}
\newacronym{ROI}{ROI}{Region Of Interest}
\newacronym{RMSE}{RMSE}{Root Mean Squared Error}
\newacronym{TSPO}{TSPO}{Translocator Protein 18 kDa}
\newacronym{TCM}{TCM}{Tissue Compartment Model}
\newacronym{SNR}{SNR}{Signal to Noise Ratio}
\newacronym{mCi}{mCi}{Millicurie}
\newacronym{VT}{V$_{\mathrm{T}}$}{Volume of Distribution}
\newacronym{IF}{IF}{Input Function}
\newacronym{NLP}{NLP}{Natural Language Processing}

% \markboth{IEEE TRANSACTIONS ON NUCLEAR SCIENCE, VOL. XX, NO. XX, XXXX 2020}
% {Author \MakeLowercase{\textit{et al.}}: Preparation of Papers for Review by the \textsc{IEEE Transactions on Nuclear  Science} \newline (May 2020)}

\usepackage{lipsum}

\begin{document}
    \title{
        \vspace{-0.75cm}
        
        A Bayesian Neural Network-Based Method for the Extraction of a Metabolite Corrected Arterial Input Function from Dynamic [$^{11}$C]PBR28 PET 
    }
    
    \author{
        \vspace{-0.25cm}
        
        Alexander~C.~Whitehead$^{*}$~\IEEEmembership{Student~Member,~IEEE},
        Ludovica~Brusaferri$^{*}$,
        Lucia~Maccioni,
        Matteo~Ferrante,
        Marianna~Inglese,
        Zeynab~Alshelh,
        Mattia~Veronese,
        Nicola~Toschi,
        Jodi~Gilman,
        Kris~Thielemans,
        and~Marco~L.~Loggia
    
        \vspace{-0.75cm}
    
        \thanks{
            \scriptsize
            This work was funded by GE Healthcare, the NIHR UCLH Biomedical Research Centre, the UCL EPSRC Centre for Doctoral Training in Intelligent, Integrated Imaging in Healthcare (i4health) grant (EP/L016478/1), the Open Source Imaging Consortium (OSIC), the NIH grants R01-NS094306-01A1, R01-NS095937-01A1 and R01-DA047088-01, the Italian Ministry of University and Research (MUR), National Recovery and Resilience Plan(NRRP), project MNESYS (PE0000006)(to NT) – A Multiscale integrated approach to the study of the nervous system in health and disease (DN. 1553 11.10.2022), National Center for HPC, BIG DATA AND QUANTUM COMPUTING (Project no. CN00000013 CN1), the PNRR National Grant DIGITAL LIFELONG PREVENTION (Project no PNC0000002DARE), and by Wellcome Trust Digital Award (no. 215747/Z/19/Z).
        }
        \thanks{
            \scriptsize
            Alexander~C.~Whitehead is with the Department of Computer Science, University College London, London, UK. Ludovica~Brusaferri, Zeynab~Alshelh, Jodi~Gilman, Nicola~Toschi, and Marco~L.~Loggia are with Athinoula A. Martinos Center for Biomedical Imaging, Harvard Medical School, Boston, MA, US. Lucia~Maccioni and Mattia~Veronese are with the Department of Information Engineering, University of Padua, Padua, Italy and Neuroimaging Department, IoPPN, King’s College London, London, UK. Matteo~Ferrante, Marianna~Inglese, and Nicola~Toschi are with the Department of Biomedicine and Prevention, University of Rome Tor Vergata, Rome, Italy.
        }
    }
    
    \pagestyle{plain}
    \pagenumbering{gobble}
    
    \maketitle
    
    \begin{abstract}
        In Positron Emission Tomography (PET), arterial sampling and metabolite correction are prerequisites for the gold-standard measurement of values like the volume of distribution ($\mathrm{V_T}$), often necessary for the full quantification of radioligand binding. However, the invasiveness and technical demands of these procedures limit their application in both research and clinical PET studies. Machine learning approaches have been explored to predict $\mathrm{V_T}$ from PET images, but their integration in clinical routine is limited by their lack of transparency or thorough evaluation. Here we propose a Bayesian Neural Network to estimate the arterial input function (AIF), while also outputting its prediction uncertainty, 1) directly from the entire dynamic PET images (NN-AEIF), 2) from an image-derived input function (IDIF) (NN-IDIF) and, as a sensitivity measure, 3) from the un-corrected plasma curve (NN-AIF). All methods, applied on [$^{11}$C]PBR28 PET data, were compared to the metabolite-corrected AIF in terms of $\mathrm{V_T}$, and the prediction uncertainty was assessed in terms of normalised coefficient of variance (nCV). Overall, both NN-AEIF and NN-AIF were able to accurately predict $\mathrm{V_T}$, outperforming the other methods, with NN-AEIF showing the lowest nCV.
    \end{abstract}
    
    % \begin{IEEEkeywords}
      %   Arterial Input Function Estimation, Signal Extraction, Dynamic PET, Machine Learning, Bayesian Neural Networks
     % \end{IEEEkeywords}
    
    \vspace{-0.5cm}

\section{Introduction} \label{sec:introduction}\vspace{-0.3cm}\vspace{0.2cm}
    \IEEEPARstart{T}{he} \gls{VT} estimated with an \gls{AIF} is utilised for quantification of many \gls{PET} tracers, including \gls{PBR28}. This, however, requires the concurrent measurement of the concentrations of unchanged radioligand in arterial plasma. Although insertion of an arterial catheter rarely results in clinically relevant adverse events, it is an invasive and laborious procedure. 
    
    \gls{IDIF} represents a promising alternative to arterial sampling~\cite{Zanotti-Fregonara2011}. However, its applicability in clinical research is hampered by several factors including the inaccuracy in the estimation of both shape and amplitude of the \gls{IF}; moreover \gls{IDIF} does not allow for radio-metabolites quantification~\cite{Sari2018Non-invasive11C-SB201745}. The application of \gls{ML} is expected to improve the accuracy of predicting the \gls{AIF} from \gls{PET} images~\cite{Kuttner2020, Ferrante2022PhysicallyImaging}. While these methods have shown promising results, the vast majority of these approaches have been developed for \gls{PET} tracers that do not produce radio-metabolites. Furthermore, even if the developed model shows sufficient prediction accuracy for unseen data, its applicability in the clinical setting remains questionable because of a lack of transparency or thorough evaluation~\cite{Salahuddin2022TransparencyMethods}. Bayesian networks offer the significant advantage of making probabilistic predictions based on available evidence. Specifically, a Bayesian network would output uncertainty estimates in addition to the model prediction. For this reason, they have the potential to overcome the key barrier to the responsible adoption of \gls{AI} in clinical practice~\cite{Prabhudesai2023LoweringAI}. 
    
    Here, we propose a Bayesian \gls{NN}-based method for predicting a metabolite corrected \gls{AIF}, while allowing for the estimation of uncertainty of the model's output.% Specifically for the \gls{AE}, although also present in the other networks, we try to enforce the low dimensional representation of the input data as disentangled and continuous. Furthermore, the network does not predict a single signal for each input; rather, it predicts a probability density function of potential signals, which allows for the estimation of uncertainty of the model's output.

    \input{methods}
    \input{results}
    \input{discussion_and_conclusion}
    
    \vspace{-0.5cm}
    
    \AtNextBibliography{
        \scriptsize
    }
    \printbibliography
\end{document}
